%!TEX root = main.tex
\section{Simulation}
\subsection{Variance reduction}
Because the simulation can be time consuming to run we want to lower the variance of the estimated parameters as much as possible relative to the computational resources used. 

\subsubsection{Control Variates}
Control variates is a variance reduction technique which can be used to reduce the variance of the estimated expected value $\hat{\mu} =E[Y]$. It works by drawing a variable $x_i$ such that $Y_i \approx f(x_i)$. One can then estimate $\hat{\mu} =E[Z]$ where $Z$ is given by

\begin{align}
Z_i = Y_i - c (X_i - E[X]) 
\end{align}

. Choosing $c = \dfrac{Cov(X, Y)}{Var(X)}$ we expect a variance reduction of

\begin{equation}
\frac{Var(E[Z])}{Var(E[Y])} = \frac{ Var(\frac{1}{N} \sum_i Z_i) }{ Var(\frac{1}{N} \sum_i Y_i )} = 1 - \rho_{x, y}^2
\end{equation}

. Since we don't have analytical expressions for the covariance or variance of our random variables we will use the approach of

\begin{enumerate}
	\item Draw $N_0$ samples $(X_{0,i}, Y_{0, i})$ and calculate $\hat{c}$
	\item Draw $N$ samples $(X_i, Y_i)$ and estimate $E[Z] = \frac{1}{N} \sum_i \left( Y_i - \hat{c} (X_i - \frac{1}{N}\sum_i X_i) \right)  $
\end{enumerate}

.