%!TEX root = main.tex

\section{Discussion}
\subsection{Control Variates}
When using control variates for estimating the peak number of infected individuals it seems like we severely underestimate the actual variance. After having spent quite some time ensuring that our code doesn't contain errors we've tried to implement a small control variate example in R trying to estimate $\int_0^1 exp(x) dx$. Even in this simply example the estimated control variate variance is way too small. We've had a TA briefly look at our R code and results without him being able to explain our observations. So we don't have a good explanation for the results.

\subsection{Effect of Olympic Games}
As expected when the Olympic Games are included in our model we saw that the virus spreads faster and thus has a higher peak infection count. Furthermore this could also be confirmed visually from our animations in which simulations with the Olympic Games showed high activation across the global at the same time. In simulations without the games it was much more apparent how the virus first spread to regions in the following order:
\begin{enumerate}
	\item South and Latin America
	\item  East coast of the USA and then Europe
	\item Africa
	\item Asia
	\item Australia
\end{enumerate}

We think this ordering is sensible because some regions are probably more heavily connected than others. E.g. when we start the virus outbreak in Rio de Janeiro but don't hold the Olympics the virus naturally  first spreads to nearby regions and to regions with which heavy plane traffic is shared.
